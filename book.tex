\documentclass[a4paper,12pt]{amsart}

\usepackage{osnova}
\usepackage{wrapfig}
\usepackage{fancyhdr}
\pagestyle{fancy}
\pagenumbering{gobble}
\fancyhead{}
\chead{Университет Иннополис. 6 класс. 2016-2017 г.}


\fancypagestyle{firststyle}{\fancyfootoffset[R]{-12cm} \renewcommand{\headrulewidth}{0.1 mm}}

\begin{document}
\fontsize{14}{16pt}\selectfont
\thispagestyle{firststyle}

\begin{center}
  {\bf Серия 1. Комбинаторика.}
  
  {\small\bf 22 сентября}
\end{center}\vspace{5mm}
\resetz

{\bf Правило сложения:} если некоторый объект $A$ можно выбрать $n$ способами, а другой объект $B$ можно выбрать $m$ способами, 
то выбор "либо $A$, либо $B$"\ можно осуществить $n + m$ способами.

\vspace{5mm}

{\bf Правило умножения:} если объект $A$ можно выбрать $n$ способами и если после каждого такого выбора объект $B$ можно выбрать 
$m$ способами, то выбор пары $(A, B)$ в указанном порядке можно осущестить $n\cdot m$ способами.

\vspace{10mm}

\zad{Бросают игральную кость с $6$ гранями и запускают волчок, имеющий $8$ граней. Сколькими различными способами могут они упасть?}

\zad{На ферме есть $20$ овец и $24$ свиньи (все разной упитанности). Сколькими способами можно выбрать одну овцу и одну свинью? 
Если такой выбор уже сделан, сколькими способами можно сделать его еще раз?}

\zad{В языке аборигенов далекого острова $10$ прилагательных, $20$ существительных и $15$ глаголов. Предложением называется 
всякое сочетание либо существительного и глагола, либо прилагательного, существительного и глагола (порядок слов в предложении 
всегда именно такой). Сколько всего предложений в этом языке?}


\zad{Король решил выдать замуж трёх своих дочерей. Со всех концов света явились во дворец сто юношей. Сколькими способами дочери 
короля могут выбрать себе женихов? }

\zad{Сколько пятибуквенных слов можно составить из $33$ букв, если допускаются повторения, 
но никакие две соседние буквы не должны совпадать, т.е. такие слова, как «пресс» или «ссора», не допускаются?}

\zad{В алфавите племени Мумбо–Юмбо всего четыре буквы: А, Е, У и О. Словом считается любая последовательность букв, 
содержащая не более, чем 3 буквы. Сколько слов в языке племени Мумбо–Юмбо?}

\zad{Сколькими способами из $28$ костей домино можно выбрать две кости так, чтобы их можно было приложить друг к другу 
(то есть чтобы какое-то число очков встречалось на обеих костях)?}

\zad{В классе учатся 10 учеников. Сколькими способами мы можем разбить класс на математиков, программистов и робототехников так, 
чтобы на каждый кружок ходил хотя бы один человек и никто не ходил одновременно на два кружка?}


\newpage

\begin{center}
  {\bf Серия 2. Круги Эйлера.}
  
  {\small\bf 25 сентября}
\end{center}\vspace{5mm}
\resetz

Каждый {\bf круг Эйлера} обозначает множество объектов, а точка -- один объект.
Точка рисуется внутри круга, если объект принадлежит этому множеству, а иначе -- снаружи круга.

\vspace{5mm}

В случае, если объект принадлежит сразу нескольким множествам, обозначающая его точка находится 
в пересечении соответствующих этим множествам кругов (то есть в каждом из них).

\vspace{5mm}

\zad{На доске нарисованы два круга, внутри которых отмечено несколько точек. Внутри первго из них всего 
$140$ отмеченных точек. Внутри второго -- всего $200$ отмеченных точек. Внутри обоих кругов одновременно 
находится ровно $90$ точек. А сколько отмеченных точек всего?}

\zad{В киоске около школы продается мороженое двух видов: пломбир и крем-брюле. 
На перемене $36$ учеников успели купить мороженое. При этом $25$ из них купили пломбир, а $17$ – мороженое крем-брюле. Сколько человек 
купили мороженое обоих сортов?}

\zad{В классе $35$ учеников. Из них $20$ занимаются греко-римской борьбой, $11$ ходят плавать в бассейн, а $10$ ребят не занимаются спортом. 
Сколько учеников ходят на обе секции?}

\zad{В кондитерском отделе супермаркета посетители обычно покупают либо один торт, либо одну коробку конфет, 
либо один торт и одну коробку конфет. В один из дней было продано $57$ тортов и $36$ коробок конфет. Сколько было покупателей, 
если $12$ человек купили и торт, и коробку конфет?}


\zad{Из $100$ школьников гимназии №6 английский язык знают $28$ школьников, немецкий -- $30$, французский -- $42$, английский и немецкий -- $8$,
английский и французский -- $10$, немецкий и французский -- $5$, все три языка знают всего $3$ школьника. Сколько школьников не знают ни одного 
из трех языков?}

\zad{На прогулку пошли шестиклассники и пятиклассники. Все они были либо босиком, либо в тапочках. Шестиклассников было $24$, а босых учеников $16$. 
Обутых пятиклассников было столько же, сколько босых шестиклассников. Сколько учеников ходили на прогулку?}

\zad{Все $5800$ жителей планеты СероБуроМалиновка носят одежду либо серого, либо бурого, либо малинового, либо серо-буро-малинового цветов. 
При этом в одежде $1200$ жителей присутствует серый цвет, в одежде $3000$ -- бурый, в одежде $2100$ -- малиновый. Сколько жителей 
носят одежду малинового цвета?}

\newpage

\begin{center}
  {\bf Серия 3. Принцип Дирихле.}
  
  {\small\bf 29 сентября}
\end{center}\vspace{5mm}
\resetz

{\bf Принцип Дирихле:} если кролики рассажены в клетки, причём число кроликов больше числа клеток, 
то хотя бы в одной из клеток находится более одного кролика.

\vspace{5mm}

{\bf Обощенный принцип Дирихле:} если $n\cdot k + 1$ или более кроликов рассажены в $n$ клеток, то хотя бы 
в одной из клеток находится не менее $k + 1$ кролика.

\vspace{5mm}

При решении следующих задач полезно каждый раз понимать, кто (или что) выполняет роль «клеток», а кто (или что) роль кроликов.

\vspace{5mm}

\zad{$10$ кроликов посадили в $9$ клеток, докажите, что есть клетка в которой оказалось хотя бы два кролика.}

\zad{$10$ кроликов посадили в $4$ клетки, докажите, что есть клетка в которой оказалось хотя бы три кролика.}

\zad{В лесу растет миллион лиственниц. Известно, что на каждой из них не более $400$ $000$ иголок. Докажите, что в 
лесу найдутся по крайней мере три лиственницы с одинаковым числом иголок.}

\zad{Докажите, что среди жителей Москвы найдутся десять тысяч, празднующих день рождения в один и тот же день. (В Москве 
проживает примерно 8 миллионов человек).}

\zad{Докажите, что если на шахматную доску поставить $9$ ладей, то какие-то две ладьи бьют друг друга.}

\zad{В клетках таблицы $3\times 3$ расставлены числа $-1$, $0$, $1$. Докажите, что какие-то две из $8$ сумм по всем строкам, 
всем столбцам и двум главным диагоналям будут равны.}

\zad{Доказать, что среди любых $2017$ натуральных чисел найдутся два, разность которых делится на $2016$.}

\zad{Несколько волейбольных команд играют турнир в один круг (каждая команда играет с каждой ровно один раз). 
Докажите, что в любой момент турнира найдутся две команды, сыгравшие к этому моменту одинаковое число матчей.}

\zad{Докажите, что среди любых шести человек есть либо трое попарно знакомых, либо трое попарно незнакомых.}

\newpage


\begin{center}
  {\bf Серия 4. Введение переменной.}
  
  {\small\bf 2 октября}
\end{center}\vspace{5mm}
\resetz

\zad{Сумма трех последовательных чисел равна $123$. Найдите эти числа.}

\zad{Может ли сумма семи последовательных чисел быть равна $1007$?}

\zad{В примере на сложение двух чисел первое слагаемое меньше суммы на $2000$, а сумма больше 
второго слагаемого на $16$. Восстановите пример.}

\zad{Для приготовления замазки для деревьев берут известь, ржаную муку и масляный лак в отношении $3:2:2$. 
Сколько каждого материала надо взять в отдельности для получения $2,8$ кг замазки?}

\zad{Для приготовления бронзы беренся $17$ частей меди, $2$ части цинка и одна часть олова. Сколько надо взять 
каждого металла в отдельности для получения $200$ кг бронзы?}

\zad{Мальвина велела Буратино разделить число на $2$, а к результату прибавить $3$. Он же по
ошибке умножил число на $2$, а от полученного произведения отнял $3$. Но ответ все равно получился правильный. Какой?}

\zad{В зоомагазине продают больших и маленьких птиц. Большая птица стоит вдвое дороже маленькой. Одна дама купила $5$ 
больших птиц и $3$ маленьких, а другая -- $5$ маленьких и $3$ больших. При этом первая дама заплатила на $20$ рублей больше. 
Сколько стоит каждая птица?}

\zad{На лужайке росли $35$ жёлтых и белых одуванчиков. После того как 8 белых облетели, а $2$ жёлтых побелели, жёлтых 
одуванчиков стало вдвое больше, чем белых. Сколько белых и сколько жёлтых одуванчиков росло на лужайке вначале?}

\zad{В трёх ящиках лежат орехи. В первом ящике на $6$ кг орехов меньше, чем в двух других вместе. А во втором -- на $10$ кг
меньше, чем в двух других вместе. Сколько орехов в третьем ящике?}

\zad{Купец продаёт двух коней с сёдлами, причем цена одного седла $120$ рублей, а другого -- $25$ рублей.
Первый конь с хорошим седлом втрое дороже другого с дешёвым, а другой конь с хорошим седлом вдвое дешевле 
первого коня с дешёвым. Какова цена каждого коня?}

\newpage

\begin{center}
  {\bf Серия 5. Счет отрезков.}
  
  {\small\bf 6 октября}
\end{center}\vspace{5mm}

\resetz


\zad{Точка $M$ -- середина отрезка $AB$, а точка $N$ -- середина отрезка $MB$. Найдите отношения $AM : MN$, $BN : AM$ и $MN : AB$.}

\zad{а) Точка $K$ отрезка $AB$, равного $12$, расположена на $5$ ближе к $A$, чем к $B$. Найдите $AK$ и $BK$.

\vspace{5mm}

б) Точка $K$ отрезка $AB$, равного $a$, расположена на $b$ $(a > b)$ ближе к $A$, чем к $B$. Найдите $AK$ и $BK$.}

\zad{Точка $M$ расположена на отрезке $AN$, а точка $N$ -- на отрезке $BM$. Известно, что $AB = 18$ и 
$AM : MN : NB = 1 : 2 : 3$. Найдите $MN$.}

\zad{На прямой выбраны четыре точки $A$, $B$, $C$ и $D$, причем $AB = 1$, $BC = 2$, $CD = 4$. Чему может 
быть равно $AD$? Укажите все возможные варианты.}

\zad{На прямой взяты точки $A$, $O$ и $B$. Точки $A_1$ и $B_1$ симметричны соответственно точкам $A$ и $B$ 
относительно точки $O$. Найдите $A_1B$, если $AB_1 = 2$.}

\zad{Точка $B$ лежит на отрезке $AC$ длиной $5$. Найдите расстояние между серединами отрезков $AB$ и $BC$.}

\zad{Даны точки $A$ и $B$. Для каждой точки $M$, не совпадающей с точкой $B$ и лежащей на прямой $AB$, 
рассмотрим отношение $AM : BM$. Где расположены точки, для которых это отношение: \quad а) больше $2$; \quad б) меньше $2$?}

\zad{В деревне у прямой дороги стоят четыре избы $A$, $B$, $C$ и $D$ на расстоянии 50 метров друг от друга. 
В какой точке дороги можно построить колодец, чтобы сумма расстояний от колодца до всех четырех изб была бы наименьшей?}

\zad{В деревне $A$ живет $50$ школьников, в деревне $B$ живет $100$ школьников. Расстояние между деревнями $3$ километра. 
В какой точке дороги из $A$ в $B$ надо построить школу, чтобы суммарное расстояние, проходимое всеми школьниками, 
было как можно меньше?}

\newpage

\begin{center}
  {\bf Серия 6. Среднее арифметическое.}
  
  {\small\bf 9 октября}
\end{center}\vspace{5mm}
\resetz

{\bf Среднее арифметическое} чисел $a_1$, $a_2,\ \ldots,\ a_n$ равно $$\overline{a} = \frac{a_1 + a_2 + \ldots + a_n}{n}$$.

\zad{Сто яблок вместе весят $5$ кг. Сколько весит «в среднем» одно яблоко? Сколько примерно весят $17$ яблок?}

\zad{В классе $10$ девочек и $20$ мальчиков. Средний рост девочки -- $140$ см, средний рост мальчика -- $149$ см. 
Найдите средний рост ученика в классе.}

\zad{В магазин привезли три сорта конфет с разной ценой: $4$ кг по цене $40$ рублей за килограмм, $3$ кг по цене $60$ 
рублей за килограмм и $1$ кг по цене $120$ рублей за килограмм. По какой цене надо продавать смесь этих конфет?}

\zad{Может ли среднее арифметическое $35$ целых чисел равняться $6,35$?}

\zad{Каждый из десяти судей оценил выступление фигуриста, и средняя оценка оказалась равна $4,2$ балла. Согласно правилам, 
были отброшены самая большая из поставленных оценок -- $6$ баллов и самая маленькая -- 2 балла, после чего опять подсчитали 
средний балл. Чему он равен?}

\zad{Когда в комнату вошел четвёртый человек, средний возраст находящихся в ней людей увеличился с $11$ до $14$ лет. 
Сколько лет вошедшему?}

\zad{Профессор Тестер проводит серию тестов, на основании которых он выставляет испытуемому средний балл. Закончив отвечать, 
Джон понял, что если бы он получил за последний тест $97$ баллов, то его средний балл составил бы $90$, а если бы он получил 
за последний тест всего $73$ балла, то его средний балл составил бы $87$. Сколько тестов в серии профессора Тестера?}

\zad{Смешали четыре раствора, содержание соли в которых составляло $10\%$, $20\%$, $30\%$ и $40\%$. При этом в одну ёмкость 
было слито $10$ г первого раствора, $20$ г второго, $30$ г третьего и $40$ г четвёртого. Каково процентное содержание соли 
в полученном растворе?}

\zad{Желая найти среднюю годовую оценку по математике у всех шестиклассников, завуч попросил учителей математики четырёх 
шестых классов вычислить средние оценки в каждом из классов и затем нашёл среднее арифметическое этих четырёх чисел. Правильно 
ли сделал завуч?}



\newpage

\begin{center}
  {\bf Серия 7. Доказательство от противного.}
  
  {\small\bf 13 октября}
\end{center}\vspace{5mm}

\resetz


{\bf Доказательство от противного} -- вид доказательства, при котором «доказывание» некоторого суждения 
(задачи или теоремы) осуществляется через опровержение {\bf отрицания} этого суждения.

\vspace{5mm}

\zad{В школе $30$ классов и $1000$ учащихся. Докажите, что есть класс, в котором не менее $34$ учеников.}

\zad{На контрольной работе учитель дал пять задач и ставил за контрольную оценку, равную количеству решённых задач. 
Все ученики, кроме Наума, решили одинаковое число задач, а Наум -- на одну больше. Первую задачу решили $9$ человек, 
вторую -- $7$ человек, третью -- $5$ человек, четвёртую -- $3$ человека, пятую -- один человек. Докажите, что Наум не мог получить
пять на контрольной?}

\zad{Можно ли в прямоугольной таблице $5\times 10$ так расставить числа, чтобы сумма чисел каждой строки равнялась бы 
$30$, а сумма чисел каждого столбца равнялась бы $10$?}

\zad{По кругу лежит $15$ шариков двух цветов. Докажите, что найдутся два соседних шарика одного цвета.}

\zad{За круглым столом сидят $25$ мальчиков и $25$ девочек. Докажите, что у кого-то из сидящих за столом оба соседа -- мальчики.}

\zad{Натуральные числа от $1$ до $23$ выписали в ряд, некоторым образом переставили, а затем от каждого числа отняли номер 
места, на котором оно стоит. Могли ли все получившиеся разности оказаться нечётными числами?}

\zad{На шахматной доске стоят $44$ ферзя. Докажите, что каждый из них бьёт какого-нибудь другого ферзя.}

\zad{Можно ли расставить на шахматной доске $17$ королей так, чтобы они не били друг друга?}

\zad{Илья захотел вписать в каждую клетку таблицы $5\times 8$ по одной цифре таким образом, чтобы каждая цифра встречалась 
ровно в четырёх рядах. (Рядами мы считаем как столбцы, так и строчки таблицы.) Потратив на это бесполезное занятие неделю,
он осознал, что это невозможно. Решите и вы эту задачу!}


\newpage

\begin{center}
  {\bf Серия 8. Разнобой.}
  
  {\small\bf 16 октября}
\end{center}\vspace{5mm}

\resetz

\zad{В группе из $16$ детей $7$ родились в Москве, $4$ -- в Санкт-Петербурге, $3$ -- в Киеве и $2$ -- в Минске. 
Сколькими способами можно выбрать из них $4$ детей так, чтобы в группе были уроженцы всех $4$ городов?}

\zad{Из двух сплавов, содержащих $10\%$ и $50\%$ меди соответственно, требуется получить новый сплав. 
В каком отношении (по массе) требуется взять исходные сплавы, чтобы получить сплав, содержащий \quad а) $30\%$; \quad б) $40\%$ меди?}

\zad{Имеются три волчка с $6$, $8$ и $10$ гранями соответственно. Их одновременно запустили. Сколькими различными 
способами они могут упасть? Сколько среди них способов, при которых по крайней мере два волчка упали на 
сторону, помеченную цифрой $1$?}

\zad{а) Доказать, что среди любых трех натуральных чисел всегда найдется два, сумма которых делится на два.

б) Докажите, что среди любых $11$ целых чисел найдутся два, разность которых делится на $10$ (разность 
которых оканчивается нулем).}

\zad{В соревновании участвовали $50$ стрелков. Первый выбил $60$ очков; второй -- 80; третий -- среднее арифметическое 
очков первых двух; четвёртый -- среднее арифметическое очков первых трёх. Каждый следующий выбил среднее 
арифметическое очков всех предыдущих. Сколько очков выбил $42$-й стрелок? А $50$-й?}

\zad{Два прямоугольника имеют равные основания. Высота первого прямоугольника равна $15$ см, а высота второго 
$8$ см. Найдите основания прямоугольников, если площадь первого на $35$ см$^2$ больше площади второго прямоугольника.}

\zad{В классе $41$ ученик. В контрольной работе Федор сделал 13 ошибок, а остальные -- меньше. 
Докажите, что, по крайней мере, четверо из них сделали ошибок поровну.}

\zad{У Аси, Баси и Васи $42$ яблока. Причем у Васи больше всех, а у Аси в $3$ раза меньше, чем у Баси. 
При этом у Баси больше, чем у Аси и Васи в среднем. Сколько яблок у каждого.}

\newpage

\begin{center}
  {\bf Серия 9. Комбинаторика 2.}
  
  {\small\bf 27 октября}
\end{center}\vspace{5mm}
\resetz

\zad{На вершину горы ведут пять дорог. Сколькими способами турист может подняться на гору и потом 
спуститься с нее? Решите ту же задачу при дополнительном условии, что подъем и спуск происходят 
по разным дорогам.}

\zad{В магазине продаются чашки пяти видов, блюдца трех видов и ложки четырех видов. Сколькими 
способами можно выбрать себе \quad а) чашку, блюдце и ложку; \quad б) два разных предмета?}

\zad{На шахматную доску надо поставить короля и ферзя. Сколькими способами это можно сделать, 
если короля надо поставить на белое поле, а ферзя — на черное? А если на цвет полей нет ограничений? 
А если обе фигуры надо поставить на белые поля?}

\zad{В корзине лежат $12$ яблок и $10$ апельсинов. Амир выбирает из нее яблоко или апельсин, после чего 
Федор берет и яблоко, и апельсин. В каком случае Федор имеет большую свободу выбора: если Амир взял 
яблоко или если он взял апельсин?}

\zad{Сколько существует пятизначных чисел, в записи которых встречаются только нечетные цифры, 
причем каждая цифра встречается ровно один раз?}

\zad{ a) В заборе $5$ досок. Каждую доску надо покрасить в синий, зелёный или жёлтый цвет, 
причём соседние доски должны быть покрашены в разные цвета. Сколькими способами это можно сделать? 

б) А если нужно, чтобы хотя бы одна из досок была синей? }

\zad{Сколько различных слов можно составить из букв слова \quad а) МАГИЯ; \quad б) ШАБАШ; \quad в) КОМБИНАТОРИКА}

\zad{В футбольной команде $11$ человек. Сколькими способами можно выбрать \quad а) капитана и заместителя; 
\quad б) двух нападающих; \quad в) трех полузащитников?}


\newpage

\begin{center}
  {\bf Серия 10. Взвешивания.}
  
  {\small\bf 30 октября}
\end{center}\vspace{5mm}

\resetz

\zad{а) Как найти из $3$ монет одну фальшивую за одно взвешивание на чашечных весах без гирь, если она легче настоящей?

б) Как найти из $8$ монет одну фальшивую за два взвешивание на чашечных весах без гирь, если она тяжелее настоящей?}

\zad{Имеется три пары монет. В каждой паре ровно одна фальшивая монета и ровно одна настоящая, причем фальшивые монеты 
легче настоящих.

а) За два взвешивания на весах без гирь разделить монеты на две группы так, чтобы в одной из групп лежали все 
настоящие, а в другой -- все фальшивые.

б) Тоже что и в а), но еще определить в какой именно группе фальшивые монеты, а в какой все -- настоящие.}

\zad{Семь монет расположены по кругу. Известно, что какие-то четыре из них, идущие подряд, -- фальшивые и что каждая фальшивая монета легче настоящей. 
Объясните, как найти две фальшивые монеты за одно взвешивание на чашечных весах без гирь.}

\zad{Имеются $25$ монет, из которых одна -- фальшивая. За два взвешивания на весах без гирь определите, что легче: 
фальшивая монета или настоящая.}

\zad{Дано $27$ монет, из которых одна фальшивая, причем известно, что фальшивая монета легче настоящей. 
Как за $3$ взвешивания на чашечных весах без гирь определить фальшивую монету?}

\zad{Имеются весы, с помощью которых можно узнать суммарный вес любых двух гирь. 

а) Как за $3$ взвешивания найти суммарный вес трех гирь?

б) Как за $7$ взвешиваний найти суммарный вес $11$ гирь?}

\zad{В казначействе лежат четыре мешка с монетами. В одном мешке все монеты фальшивые (весом 9г), 
а в трех других -- все настоящие (весом 10г). Как за одно взвешивание на весах со стрелкой определить, где какие монеты?}

\zad{В казначействе лежат четыре мешка с монетами. В некоторых мешках все монеты фальшивые (весом 9г), 
а во всех остальных -- все настоящие (весом 10г). Как за одно взвешивание на весах со стрелкой определить, где какие монеты?}

\newpage

\begin{center}
  {\bf Серия 11. Четность.}
  
  {\small\bf 3 ноября}
\end{center}\vspace{5mm}
\resetz

\zad{Можно ли разменять $25$ рублей при помощи десяти купюр достоинством в $1$, $3$ и $5$ рублей?}

\zad{Наум говорит, что знает четыре числа, сумма и произведение которых -- нечетные числа. Прав ли Наум?}

\zad{По кругу написано $11$ натуральных чисел. Верно ли, что среди этих чисел найдутся два соседних, сумма которых четна?}

\zad{На доске написано равенство: $1*2*3*4*5*6*7*8*9=10$ (вместо символов «*» -- в неизвестном порядке расставлены знаки «+» и «-»). 
Докажите, что это равенство не может быть верным. }

\zad{На чудо-дереве росли $30$ апельсинов и $25$ бананов. Каждый день садовник снимал ровно два фрукта. 
Причем, если он снимал одинаковые фрукты, то на дереве появлялся новый банан, а если разные -- новый апельсин. 
В конце концов, на дереве остался один фрукт. Какой: банан или апельсин?}

\zad{Конь вышел из поля $a1$, сделал несколько ходов и вернулся в то же место. Докажите, что он сделал четное количество ходов}

\zad{На доске записано число $123456789$. У написанного числа выбираются две соседние цифры, 
если ни одна из них не равна $0$, из каждой цифры вычитается по $1$, и выбранные цифры меняются 
местами (например, из $123456789$ можно за одну операцию получить $123436789$). Какое наименьшее 
число может быть получено в результате таких операций? }

\newpage

\begin{center}
  {\bf Серия 12. Среднее арифметическое 2.}
  
  {\small\bf 10 ноября}
\end{center}\vspace{5mm}
\resetz


\zad{На сколько уменьшится средний возраст команды из $11$ футболистов, если закончившего выступление 
$32$-летнего игрока заменит игрок в возрасте $21$ год?}

\zad{Из команды ушёл баскетболист ростом $192$ см, при этом средний рост команды не изменился. Чему он мог быть равен?}

\zad{Компания друзей детства встретилась через 10 лет. Как за это время изменился средний возраст компании?}

\zad{В школе было решено перейти с пятибалльной системы оценок на $40$-балльную. Для этого каждую текущую оценку 
ученика умножили на $8$. Как изменился средний балл ученика?}

\zad{Баскетболист Джон перешёл из одной команды в другую. Мог ли в обеих командах вырасти средний рост?}

\zad{В последнюю неделю за любые три дня подряд Робин-Бобин в среднем съедал по $10$ пончиков в день. 
Верно ли, что за эту неделю он в среднем съел $10$ пончиков в день?}

\zad{Боб и Ваня соревнуются в изготовлении и употреблении сладких коктейлей. Боб смешал «пепси» с «Фантой», 
а Ваня -- лимонад с сиропом. Известно, что лимонад слаще «пепси», а сироп слаще «фанты». Могла ли смесь 
Боба оказаться слаще Ваниной? (Сладость -- это доля сахара от общего веса.)}

\zad{По окружности расставлены $100$ чисел так, что каждое из них равно среднему арифметическому двух своих соседей. 
Докажите, что все числа между собой равны.}


\newpage

\begin{center}
  {\bf Серия 13. Введение переменной 2.}
  
  {\small\bf 13 ноября}
\end{center}\vspace{5mm}
\resetz


\zad{Имеется семь последовательных натуральных чисел. Сумма первых трех равна $33$. 
Чему равна сумма последних трех?}

\zad{После того, как бегун пробежал треть всей дистанции и еще $400$ м, ему осталось пробежать 
треть пути и еще $200$ м. Чему равна длина дистанции?}

\zad{В классе $30$ учеников. Они сели за парты по двое так, что каждый мальчик сидит с девочкой, 
и ровно половина девочек сидит с мальчиками. Сколько в классе мальчиков?}

\zad{Придя в магазин, Винни-Пух обнаружил, что горшочек для меда подорожал на $50\%$, а
мед подешевел на $50\%$, и теперь горшочек и мед в нем стоят поровну. Как изменилась
цена горшочка с медом?}

\zad{У Пети и Коли были две одинаковые прямоугольные карточки. Каждый
мальчик разрезал свою карточку на два прямоугольника. Сумма периметров прямоугольников, 
которые получились у Пети, равна $40$, а у Коли -- $50$. Чему равен периметр исходной
карточки? }

\zad{Капитан Флинт и несколько пиратов нашли сундук с золотыми монетами. Они
разделили монеты поровну. Если бы пиратов было на 4 меньше, то каждый получил бы на $10$ 
монет больше. Если бы монет было на $50$ меньше, то каждый пират получил бы на пять
монет меньше. Сколько золотых монет было в сундуке? }

\zad{Два туриста вышли одновременно из села $A$ в село $B$. Когда первый турист 
прошел половину пути, второму осталось пройти $24$ км, а когда второй прошел половину пути, 
первому осталось пройти $15$ км. Каково расстояние между $A$ и $B$?}


\newpage

\begin{center}
  {\bf Серия 14. Можно или нельзя.}
  
  {\small\bf 17 ноября}
\end{center}\vspace{5mm}

\resetz

\zad{Можно ли числа $1,\ 2,\ 3,\ \ldots,\ 100$ разбить на пары (четное, нечетное) так, чтобы во всех парах, 
кроме одной, нечетное число было больше четного?}

\zad{Карабас-Барабас принёс Джузеппе лист фанеры размером $10\times 10$ и хочет, чтобы Джузеппе распилил без 
остатка этот лист на прямоугольные заготовки размером $1\times 3$. Браться ли Джузеппе за эту работу?}

\zad{Можно ли из полосок $1\times 1, 1\times 2, \ldots, 1\times 13$ сложить прямоугольник со сторонами больше $1$?}

\zad{Может ли у прямоугольника площади $1$ квадратный сантиметр быть периметр быть больше $1$ километра?}

\zad{На плоскости нарисован черный квадрат. Имеется семь квадратных плиток того же размера. Можно ли положить их на 
плоскость так, чтобы они не перекрывались и чтобы каждая плитка покрывала хотя бы часть черного квадрата (хотя бы 
одну точку внутри него)?}

\zadris{Художник-авангардист Змий Клеточкин покрасил несколько клеток доски размером $7\times 7$, соблюдая правило: 
каждая следующая закрашиваемая клетка должна соседствовать по стороне с предыдущей закрашенной клеткой, но не 
должна соседствовать ни с одной другой ранее закрашенной клеткой. Ему удалось покрасить $31$ клетку. Можно ли 
побить его рекорд.}{0.7}{2}

\zad{Числитель увеличили на $1$, а знаменятель на $100$. Могла ли дробь увеличиться?}

\zad{Сможете ли вы написать несколько целых положительных чисел таких, чтобы их сумма была в $5$ раз больше их произведения?}


\newpage

\begin{center}
  {\bf Серия 15. Счет углов.}
  
  {\small\bf 20 ноября}
\end{center}\vspace{5mm}

\resetz


{\bf Аксиома:} Градусная мера развернутого угла равна $180^{\circ}$.

\vspace{5mm}

{\it Смежные углы} -- это пара углов с общей вершиной и одной общей стороной. 
Две другие стороны составляют продолжение одна другой и образуют прямую линию. 
Таким образом, вместе смежные углы составляют развёрнутый угол. Поэтому сумма величин смежных углов $180^{\circ}$.

\vspace{5mm}

{\bf Равенство вертикальных углов:} Прямые $AB$ и $CD$ пересекаются в точке $O$. Тогда $\angle AOC = \angle BOD$ и $\angle AOD = \angle BOC$. 
Эти углы называются {\it вертикальными}. 

\vspace{5mm}

{\it Биссектриса угла} -- луч с началом в вершине угла, делящий угол на два равных угла.

\resetz

\zad{Один из двух смежных углов на $30^{\circ}$ больше другого. Найдите эти углы.}

\zad{Один из двух смежных углов в $3$ раза меньше другого. Найдите эти углы.}

\zad{Докажите, что биссектрисы двух смежных углов перпендикулярны.}

\zad{Докажите, что биссектрисы двух вертикальных углов лежат на одной прямой.}

\zad{Точка $M$ лежит внутри $\angle AOB$, $OC$ -- биссектриса этого угла. Докажите, что $\angle MOC$ равен полуразности $\angle AOM$ и $\angle BOM$.}

\zad{а) На сколько градусов поворачивается за минуту минутная стрелка? Часовая стрелка?

б) Какой угол образуют минутная и часовая стрелка в 3 часа 5 минут?

в) В полдень минутная и часовая стрелка совпали. Когда они совпадут в следующий раз?

г) Сколько раз в течение суток часовая и минутная стрелки совпадают?}


\newpage

\begin{center}
  {\bf Серия 16. Круги Эйлера 2.}
  
  {\small\bf 24 ноября}
\end{center}\vspace{5mm}

\resetz


\zad{Только на кружок математики ходят $5$ учеников класса, а только на робототехнику $8$ учеников, причём каждый где-то занимается. 
Сколько учеников занимаются и там и там, если в классе учится $22$ человека?}

\zad{В саду были срезаны тюльпаны: белые и розовые -- $400$ штук, розовые и красные -- $300$ штук, 
белые и красные -- $440$ штук. Сколько всего тюльпанов было срезано в саду?}

\zad{В спортзале площадью $12$ кв. м лежат три мата. Площадь первого мата $5$ кв. м, второго -- $4$ кв. м, третьего -- $3$ кв. м. 
Каждые два мата перекрываются на площади $1,5$ кв. м. Все три мата перекрываются на площади $0,5$ кв. м. 
Какова площадь зала, не покрытого матами?}

\zad{Сколько детей в семье, если $7$ из них любят капусту, $6$ -- морковь, $5$ -- горох, $4$ -- капусту и морковь, 
$3$ -- капусту и горох, $2$ -- морковь и горох, а $1$ любит капусту, и горох, и морковь? }

\zad{Сколько существует целых положительных чисел, меньших 100, которые 

a) делятся на $3$; \quad б) делятся на $5$; \quad в) делятся и на $3$ и на $5$; 

г) делятся на $3$, но не делятся на $5$; д) делятся на $5$, но не делятся на $3$;

е) не делятся ни на $3$, ни на $5$.}

\zad{Четверо друзей заметили, что если они сложатся без первого, то соберут $90$ рублей,
без второго -- $85$, без третьего -- $80$, без четвёртого -- $75$ рублей. Сколько денег у всех четверых вместе?}

\zad{На собрании клуба ветеранов "Веселый Роджер"\ присутствовало $100$ человек. $80$ из них были без уха, $85$ -- без глаза, а $75$ -- без носа. 
Каково минимальное возможное количество ветеранов, обладающих всеми тремя увечьями?}


\newpage

\begin{center}
  {\bf Серия 17. Десятичная запись.}
  
  {\small\bf 27 ноября}
\end{center}\vspace{5mm}
\resetz

\zad{Наум пишет подряд натуральные числа: $123456789101112\ldots $. На каких местах, считая от начала, 
в первый раз будут стоять три цифры «5» подряд? А четыре цифры $5$?}

\zad{Решите ребус: $\overline{ABBA} + A + B = \overline{CDDA}$, где одинаковые буквы обозначают одинаковые цифры, разные буквы -- разные цифры.}

\zad{Чему равно выражение: $$(\hbox{К} \times \hbox{А} \times \hbox{Р} \times \hbox{Л} \times \hbox{С} \times \hbox{О} 
\times \hbox{Н}) \times (\hbox{В} \times \hbox{А} \times \hbox{Р} \times \hbox{Е} \times \hbox{Н} \times \hbox{Ь} \times \hbox{Е}),$$
где разные буквы обозначают разные цифры, одинаковые буквы -- одинаковые цифры?}

\zad{Решите ребус:
$$\hbox{К} \times \hbox{О} \times \hbox{Т} = \hbox{У} \times \hbox{Ч} \times \hbox{Ё} \times \hbox{Н} \times \hbox{Ы} \times \hbox{Й},$$
где разные буквы обозначают разные цифры, одинаковые буквы -- одинаковые цифры.}

\zad{Дано трёхзначное число $\overline{ABB}$, произведение цифр которого -- двузначное число $\overline{AC}$, произведение цифр этого 
числа равно цифре $C$ (здесь, как в математических ребусах, цифры в записи числа заменены буквами; одинаковым буквам соответствуют
одинаковые цифры, разным -- разные). Определите исходное число.}

\zad{Найдите наибольшее шестизначное число, у которого каждая цифра, начиная с третьей, равна сумме двух предыдущих цифр.}

\zad{Найдите наибольшее число, у которого каждая цифра, начиная с третьей, равна сумме двух предыдущих цифр. }

\zad{Найдите все двузначные числа, которые в $5$ раз больше суммы своих цифр.}


\newpage

\begin{center}
  {\bf Серия 18. Разнобой 2.}
  
  {\small\bf 1 декабря}
\end{center}\vspace{5mm}

\resetz


\zad{У скольки пятизначных чисел все цифры различны?}

\zad{Существует ли такое десятизначное натуральное число, в котором все числа различны и при вычеркивании 
из него любых четырех цифр получится четное число?}

\zad{Через точку на плоскости провели $4$ прямых, после чего плоскость разрезали по этим прямым на углы. Докажите, что хотя 
бы один из этих углов меньше $46^{\circ}$}

\zad{Напишите в строчку первые $10$ простых чисел. Как вычеркнуть $6$ цифр, чтобы получилось наибольшее возможное число?}

\zad{Саша и его папа собирали грибы. Саша нашел на $18$ грибов больше, чем половина грибов, найденных папой. 
Папа нашел на $7$ грибов больше, чем Саша. Сколько грибов нашли Саша и папа вместе?}

\zad{Внутри угла $AOC$ провели луч $OB$. Лучи $OM$ и $ON$ -- биссектрисы углов $AOB$ и $BOC$ соответсвенно. Докажите, что угол $MON$ 
в два раза меньше угла $AOC$.}

\zad{Цифры $1$, $2$, $\ldots$, $9$ разбили на три группы. Докажите, что произведение чисел хотя бы в одной группе меньше $72$.}

\zad{Существуют ли такие двузначные числа $\overline{ab}$ и $\overline{cd}$, что $\overline{ab} \cdot \overline{cd} = \overline{abcd}$.}

\zad{Сколько шестибуквенных слов можно составить из $32$ букв, содержащих хотя бы один раз букву «а»?}

\newpage

\begin{center}
  {\bf Серия 19. Обратный ход.}
  
  {\small\bf 4 декабря}
\end{center}\vspace{5mm}
\resetz

\zad{Наум задумал число. Потом прибавил к нему $5$, разделил на $3$, умножил на $4$, отнял $6$, разделил на $7$ и получил $2$. 
Какое число задумано?}

\zad{Женщина собрала в саду груши. Чтобы выйти из сада, ей пришлось пройти через четыре двери, каждую из которых 
охранял свирепый стражник, отбиравший половину груш. Домой она принесла $10$ груш. Сколько груш досталось стражникам?}

\zad{На озере расцвела одна лилия. Каждый день количество цветов на озере удваивалось, и на $20$-й день все озеро покрылось цветами. 
На какой день озеро покрылось цветами наполовину? }

\zad{Федор выписал в ряд несколько единиц. Через минуту между любыми двумя соседними единицами он написал еще по одной единице. 
Ещё через минуту проделал то же самое. В итоге в тетради у Федора оказалось 1197 единиц. Сколько их было изначально?}

\zad{Амир задумал натуральное число, умножил его на $13$, зачеркнул последнюю цифру результата, полученное число умножил на $7$, 
зачеркнул последнюю цифру результата и получил $21$. Какое число задумал Амир?}

\zad{По кругу расставлены $9$ нулей и единиц, причём не все расставленные числа равны. За один ход между каждыми двумя 
соседними числами записывается $0$, если эти числа равны, и $1$, если они не равны. После этого старые числа стираются. 
Могут ли через некоторое время все числа стать равными? }

\zad{Все натуральные числа от $1$ до $1000$ записали в следующем порядке: сначала были выписаны в порядке возрастания числа, 
сумма цифр которых равна $1$, затем, также в порядке возрастания, числа с суммой цифр $2$, потом -- числа, сумма цифр которых 
равна $3$ и т.д. На каком месте оказалось число $996$?}


\newpage

\begin{center}
  {\bf Серия 20. Разнобой 3.}
  
  {\small\bf 8 декабря}
\end{center}\vspace{5mm}

\resetz

\zad{Существуют ли такие различные натуральные числа $a$ и $b$, что $a$ делится на $b$, 
$a + 1$ делится на $b+1$ и $a + 2$ делится на $b + 2$?}

\zad{В полосе из 11 клеток стоят числа. Причем в первой клетке стоит число 6, а в девятой клетке число 4. 
Известно, что сумма чисел в любых трех подряд идущих клетках равна 15. Какое число стоит в последней клетке?}

\zad{В январе консервы из детей, плохо решающих задачи, подорожали на $20\%$, а в феврале -- подешевели на $20\%$.
На сколько процентов в итоге изменилась цена?}

\zad{На доске написана дробь $a/b$ с натуральным числителем и знаменателем. Максим умножил числитель этой дроби на 2016. 
Докажите, что Сергей может прибавить к знаменателю такое натуральное число, чтобы в итоге получилась дробь, вдвое большая, чем $a/b$.}

\zad{Илья, вернувшись из страны Врунляндии, рассказал, что из каждого озера в этой стране вытекает 3 реки, 
и в каждое озеро впадает 4 реки. Кроме того, ни одна река не вытекает за пределы Врунляндии и не втекает в нее из вне, 
каждая река начинается в каком-нибудь озере, и заканчивается в каком-нибудь (возможно другом). Докажите, что Илья соврал?}

\zad{Из трех различных (ненулевых) цифр составили всевозможные двузначные числа. Их сумма равна 132. Какие цифры были взяты?}

\zad{Существуют ли такие натуральные числа $a$ и $b$, что $a$ делится на $b$, 
$a + 1$ делится на $b+1$, $a + 2$ делится на $b + 3$ и $a+3$ делится на $b+2$?}


\newpage

\begin{center}
  {\bf Серия 21. Проценты, проценты и еще раз проценты.}
  
  {\small\bf 11 декабря}
\end{center}\vspace{5mm}
\resetz


\zad{На острове $2/3$ всех мужчин женаты и $3/5$ всех женщин замужем. Какая доля населения острова состоит в браке?}

\zad{В классе учится меньше $50$ школьников. За контрольную работу седьмая часть учеников получила пятёрки, 
третья -- четвёрки, половина -- тройки. Остальные работы были оценены как неудовлетворительные. Сколько было таких работ?}

\zad{В течение года цены на штрюдели два раза поднимали на $50\%$, а перед Новым Годом их стали продавать за полцены.
Сколько стоит сейчас один штрюдель, если в начале года он стоил $80$ рублей? }

\zad{В тесте к каждому вопросу указаны $5$ вариантов ответа. Отличник отвечает на все вопросы правильно. Когда двоечнику 
удаётся списать, он отвечает правильно, а в противном случае -- наугад (то есть среди несписанных вопросов он правильно 
отвечает на $1/5$ часть). Всего двоечник правильно ответил на половину вопросов. Какую долю ответов ему удалось списать?}

\zad{Три пирата делили мешок монет. Первый забрал $3/7$ всех монет, второй -- $51\%$ остатка, после чего третьему осталось 
на $8$ монет меньше, чем получил второй. Сколько монет было в мешке?}

\zad{На перемене несколько учащихся ушли из лицея и несколько пришли в него. В результате количество учеников в 
лицее после перемены уменьшилось на $10\%$, а доля мальчиков среди учеников лицея увеличилась с $50\%$ до $55\%$. 
Увеличилось или уменьшилось количество мальчиков?}

\zad{В банановой республике прошли выборы в парламент, в котором участвовали все жители. Все голосовавшие за партию 
"Мандарин"\ любят мандарины. Среди голосовавших за другие партии $90\%$ не любят мандарины. Сколько процентов голосов
набрала партия "Мандарин"\ на выборах, если ровно $46\%$ жителей любят мандарины?}


\newpage

\begin{center}
  {\bf Серия 22. Оценка + пример.}
  
  {\small\bf 15 декабря}
\end{center}\vspace{5mm}

\resetz


\zad{Какое наибольшее число трехклеточных уголков можно вырезать из клетчатого квадрата $8 \times 8$?}

\zad{В пять горшочков, стоящих в ряд, Кролик налил три килограмма мёда (не обязательно в каждый и не обязательно 
поровну). Винни-Пух может взять любые два горшочка, стоящие рядом. Какое наибольшее количество мёда сможет 
гарантированно съесть Винни-Пух?}

\zad{Какое наибольшее число прямоугольников $1\times 5$ можно вырезать из квадрата $8\times 8$?}

\zad{На клетчатой доске $10 \times 10$ закрасили $n$ доминошек (прямоугольников $1 \times 2$). Оказалось, что в 
каждой строке и в каждом столбце есть хотя бы одна закрашенная клетка. При каком наименьшем $n$ это возможно?}

\zad{Какое наибольшее число белых и чёрных фишек можно расставить на шахматной доске так, чтобы на любой горизонтали 
и на любой вертикали белых фишек было ровно в два раза больше, чем чёрных? (Каждая фишка занимает отдельную клетку.)}

\zad{а) На какие двузначные числа может оканчиваться четырехзначное число делящееся на $4$?

б) Какое наибольшее произведение цифр может иметь четырехзначное число, делящееся нацело на 4?}

\zad{Петя выписал по кругу в некотором порядке целые числа от $1$ до $10$ и затем отметил те из них, 
которые равны сумме двух своих соседей. Какое наибольшее количество чисел могло быть отмечено?}



\newpage

\begin{center}
  {\bf Серия 23. Знакомство с графами.}
  
  {\small\bf 18 декабря}
\end{center}\vspace{5mm}

\resetz


{\bf Граф} -- совокупность точек, некоторые из которых соединены отрезками. Точки называются вершинами графа, 
отрезки -- ребрами. 

\vspace{5mm}

\zad{В стране Цифра есть $9$ городов с названиями $1$, $2$, $3$, $4$, $5$, $6$, $7$, $8$, $9$. Путешественник 
обнаружил, что два города соединены авиалинией в том и только в том случае, если двузначное число, составленное 
из цифр-названий этих городов, делится на $3$. Можно ли добраться из города $1$ в город $9$?}

\zad{В некотором государстве $6$ городов и $10$ автодорог, каждая из которых связывает какие-то два города. 
Между городами устанавливается авиационное сообщение, исходя из принципа экономии: авиационная линия между 
двумя городами устанавливается тогда и только тогда, когда автомобильная дорога между этими городами отсутствует. 
Сколько авиалиний будет проведено?}

\vspace{5mm}

{\bf Степень вершины} -- количество ребер, выходящих из данной вершины. 

\vspace{5mm}

\zad{В государстве $100$ городов, и из каждого из них выходит $4$ дороги. Сколько всего дорог в государстве?}

\zad{Может ли в государстве, в котором из каждого города выходит $3$ дороги, быть ровно $100$ дорог?}

\zad{Докажите, что не существует графа с пятью вершинами, степени которых равны $4$, $4$, $4$, $4$, $2$. }

\zad{Наум считает, что в его классе у всех разное число друзей-одноклассников. Не ошибается ли он?}

\zad{Илья утверждает, что среди любых а) четырёх; б) пяти; в) шести человек обязательно найдётся либо 
трое знакомых друг с другом, либо трое незнакомых. Не завирается ли он?}


\newpage

\begin{center}
  {\bf Серия 24. Делимость.}
  
  {\small\bf 22 декабря}
\end{center}\vspace{5mm}
\resetz

{\bf Определение.} Целое число $a$ {\bf делится} на целое число $b \neq 0$ (или $a$ {\bf кратно} $b$, или $b$ {\bf делитель} $a$), 
если найдётся такое целое число $q$, что $a = b \cdot q$. Обозначение: $a\ \vdots\ b$.

Заметим, что если $a\ \vdots b$, то $a\ \vdots\ (-b)$. Поэтому, если не оговорено противное, мы будем искать только 
положительные делители чисел.

\vspace{5mm}

\zad{Заполните пустые клетки таблицы знаками «+», «–» и «?» интуитивно понятным
образом. Не забудьте привести доказательства.
\ris{0.6}{3}}

\zad{Определите, не выполняя действий, делится ли \quad а) $18^2 - 7^2$ на $11$; \quad б) $1 + 2 + \ldots + 82$ на $83$.}

\zad{В каком случае два числа $a$ и $b$ таковы, что $a$ делится на $b$ и $b$ делится на $a$? (Числа могут быть и отрицательными)}

\zad{Верно ли, что если $a\ \vdots\ b$ и $b\ \vdots\ c$, то $a\ \vdots\ c$?}

\zad{Верно ли, что если $a\ \vdots\ m$ и $b\ \vdots\ n$, то $ab\ \vdots\ mn$?}

\zad{Максим считает, что если $a^2$ делится на $a - b$, то $b^2$ делится на $a - b$. Прав ли он?}

\zad{Докажите, что квадрат натурального числа имеет нечётное количество делителей. б) Верно ли обратное?}


\newpage

\begin{center}
  {\bf Серия 25. Делимость 2.}
  
  {\small\bf 29 декабря}
\end{center}\vspace{5mm}
\resetz


\zad{Приведите пример числа, которое: а) делится на 5 и делится на 6; б) делится на 11 и делится на 12.}

\zad{Может ли сумма трёх различных натуральных чисел делиться на каждое из слагаемых?}

\zad{Докажите, что сумма любых пяти последовательных чисел делится на $5$.}

\zad{Дети ходили в лес за орехами и теперь, возвращаясь домой, идут парами. В каждой паре 
идут мальчик и девочка, причём у мальчика орехов в $2$ раза больше, чем у девочки. Может ли всего у детей быть $100$ орехов?}

\zad{В магическом квадрате суммы цифр в каждой строке, в каждом столбце и на обеих диагоналях равны. 
Можно ли составить магический квадрат $3\times 3$ из первых $9$ простых чисел? }

\zad{Можно ли расставить числа а) от $1$ до $7$; б) от $1$ до $9$ по кругу так, чтобы любое из них делилось на разность своих соседей? }

\zad{Докажите, что сумма a) $1 + 2 + \ldots + (2n + 1)$ делится на $2n + 1$; б) любых $2n + 1$ последовательных чисел делится на $2n + 1$.}

\newpage

\begin{center}
  {\bf Серия 26. Четность 2.}
  
  {\small\bf 12 января}
\end{center}\vspace{5mm}
\resetz

\zad{Разность двух целых чисел умножили на их произведение. Могло ли получиться число $1995$?}

\zad{Можно ли так расставить знаки в выражении $\pm 1\pm 2\pm \ldots \pm10$ чтобы получился $0$?}

\zad{Можно ли число $33! = 33 \cdot 32 \cdot \ldots \cdot 1$ представить как сумму $33$ последовательных нечетных чисел?}

\zad{Наум написал на листке число $20$. Тринадцать шестиклассников передают листок друг другу, и каждый прибавляет 
к числу, или отнимает от него единицу или тройку -- как хочет. Может ли получиться в результате число $8$?}

\zad{На каждой клетке главной диагонали шашечной доски размером $10 \times 10$ стоит по шашке. За один ход можно 
выбрать любые две шашки и передвинуть каждую из них на одну клетку вниз. Можно ли за несколько таких ходов поставить 
поставить все шашки на нижнюю горизонталь доски?}

\zad{В клетчатом квадрате, разрезая по границам клеток, прорезали квадратную дырку поменьше. Может ли оставшаяся 
фигура состоять ровно из $250$ клеток?}

\zad{Простые числа $p$ и $q$ таковы, что $3p + 5q = 511$. Найдите эти числа. Укажите все возможные варианты.}

\zad{Шестизначный номер называется почти счастливым, если сумма трёх каких-то его цифр равна сумме трёх остальных. 
Костя взял в автобусе два билета подряд. Их номера оказались почти счастливыми. Докажите, что один из этих номеров 
оканчивается на $0$.}


\newpage

\begin{center}
  {\bf Серия 27. Признаки делимости.}
  
  {\small\bf 15 января}
\end{center}\vspace{5mm}
\resetz


{\bf Признаки делимости:} \begin{itemize}
                           \item[1)] число делится на 3, если сумма всех его цифр делится на 3;
                           \item[2)] число делится на 9, если сумма всех его цифр делится на 9;
                           \item[3)] число делится на 2, если его последняя цифра делится на 2;
                           \item[4)] число делится на 4, если число, образованное последними двумя цифрами, делится на 4;
                           \item[5)] число делится на 5, если его последняя цифра делится на 5.
                          \end{itemize}


\zad{Какие цифры можно поставить на месте звёздочки в числе $67800215*33$, чтобы полученное число делилось а) на $9$ б) на $3$?}

\zad{Запишите несколько раз подряд число $4242$ так, чтобы получившееся число делилось на $9$.}

\zad{Делится ли число $10^{666} + 8$ на $9$?}

\zad{Фрекен Бок спрятала плюшки в сейф, а Карлсон с Малышом хотят их от туда достать. Они знают, что код от сейфа состоит из $7$ цифр -- двоек и троек, 
причем двоек больше, чем троек. И что код делится и на $3$, и на $4$. Смогут ли Малыш и Карлсон с первой попытки открыть сейф?}

\zad{Илья загадал $10$-значное число, кратное $9$. Чему равна сумма цифр суммы цифр этого числа?}

\zad{Амир задумал простое трёхзначное число, все цифры которого различны. На какую цифру оно может оканчиваться, если его 
последняя цифра равна сумме первых двух?}

\zad{Какие цифры можно поставить на месте звёздочек в числе $72*4*$ так, чтобы оно делилось на $45$?}

\zad{Докажите, что из любых семи различных цифр можно составить число, которое делится на 4.}


\newpage

\begin{center}
  {\bf Серия 28. Шахматная и другие расскраски.}
  
  {\small\bf 19 января}
\end{center}\vspace{5mm}
\resetz

\zad{Из прямоугольника $5\times 10$ вырезали две угловые клетки $(1, 1)$ и $(1, 5)$. Докажите, что оставшуюся часть нельзя разрезать 
на четырехклеточные уголки.} 

\zad{На каждой клетке доски $9\times 9$ сидит один дрессированный лягушонок. По команде «Ква!» каждый лягушонок перепрыгивает 
на одну из соседних клеток, (клетки считаются соседними, если они имеют общую сторону). Докажите, что после команды «Ква!» 
какие-то два лягушонка окажутся на одной клетке. }

\zad{В центре куба $3\times 3\times 3$ сидит жук. Доказать, что он, переползая через грани, не сможет обойти все кубики $1\times 1\times 1$ по одному разу.}

\zad{Можно ли выложить прямоугольник $6 \times 10$ прямоугольниками $1 \times 4$?}

\zad{Можно ли выложить квадрат $8 \times 8$, используя $15$ прямоугольников $1 \times 4$ и один четырехклеточный уголок?}

\zad{Можно ли выложить шахматную доску тридцатью двумя доминошками так, чтобы $17$ из них были расположены горизонтально, а $15$ -- вертикально?}

\zad{В каждой клетке квадрата $9\times 9$ сидит жук. По команде каждый жук перелетает на одну из соседних по диагонали клеток. 
Докажите, что по крайней мере $9$ клеток после этого окажутся свободными.}


\newpage

\begin{center}
  {\bf Серия 29. Графы 2.}
  
  {\small\bf 26 января}
\end{center}\vspace{5mm}

\resetz


{\bf Граф} -- совокупность точек, некоторые из которых соединены отрезками. Точки называются вершинами графа, 
отрезки -- ребрами. 

{\bf Степень вершины} -- количество ребер, выходящих из данной вершины.

\vspace{5mm}

{\bf Теорема 1.} Сумма степеней всех вершин графа равна удвоенному количеству всех ребер.

{\bf Теорема 2.} Число вершин нечетной степени любого графа четно.

\vspace{5mm}

\zad{В шахматном турнире по круговой системе участвуют семь школьников. Известно, что Ваня сыграл шесть партий, 
Толя -- пять, Леша и Дима по три, Семен и Илья по две, Женя -- одну. С кем сыграл Леша?}

\zad{Могут ли степени вершин в графе быть равны:

а) 8, 6, 5, 4, 4, 3, 2, 2?

б) 7, 7, 6, 5, 4, 2, 2, 1?

в) 6, 6, 6, 5, 5, 3, 2, 2?}

\zad{В соревновании по круговой системе с двенадцатью участникам и провели все встречи. Сколько встреч было сыграно?}

\zad{Существует ли граф, содержащий более одной вершины, никакие две вершины которого не имеют одинаковой степени?}

\zad{В стране 100 деревень и несколько городов. Из каждой деревни выходит 4 дороги, а из каждого города -- 5. Каждая 
дорога начинает и заканчивается в деревне или городе. Сколько в этой стране а) дорог; б) городов?}

\zad{На клетчатом листе закрасили 25 клеток. Может ли каждая из них иметь нечётное число закрашенных соседей?}

\zad{В графе каждая вершина -- синяя или зелёная. При этом каждая синяя вершина связана с 5 синими и 10 зелёными, 
а каждая зелёная -- с 9 синими и 6 зелёными. Каких вершин больше -- синих или зелёных?}


\newpage

\begin{center}
  {\bf Серия 30. Комбинаторика 3.}
  
  {\small\bf 29 января}
\end{center}\vspace{5mm}

\resetz


\zad{Имеется $6$ пар перчаток различных размеров. Сколькими способами можно выбрать из них одну перчатку 
на левую руку и одну — на правую руку так, чтобы эти перчатки были различных размеров?}

\zad{Во скольких девятизначных числах все цифры различны?}

\zad{Сколько существует семизначных телефонных номеров, в первых $3$ цифрах которых не встречаются цифры $1$ и $2$?}

\zad{На загородную прогулку поехали $92$ человека. Бутерброды с колбасой взяли $48$ человек, с сыром -- $38$ человек, 
с ветчиной -- $42$ человека, с сыром и колбасой -- $28$ человек, с колбасой и ветчиной -- $31$ человек, 
с сыром и ветчиной -- $26$ человек. Все три вида бутербродов взяли с собой $25$ человек, а остальные вместо бутербродов
взяли пирожки. Сколько человек взяли с собой пирожки?}

\zad{Сколько целых чисел от 0 до $999$ не делятся ни на $5$, ни на $7$?}

\zad{Переплетчик должен переплести $12$ различных книг в переплеты а) двух; б) трех цветов. 
Сколькими способами он может это сделать, если в каждый цвет должна быть переплетена хотя бы одна книга?}

\zad{Сколько существует шестизначных чисел, в записи которых есть хотя бы одна четная цифра?}

\zad{Сколько шестизначных чисел содержат ровно три различные цифры?}

\zad{Сколько чисел от $0$ до $999$ не делятся ни на $2$, ни на $3$, ни на $5$, ни на $7$?}



\newpage

\begin{center}
  {\bf Серия 31. Делимость 3.}
  
  {\small\bf 2 февраля}
\end{center}\vspace{5mm}
\resetz


\zad{Ковбой Джо зашел в бар и попросил у бармена бутылку виски за $3$ доллара, трубку за шесть долларов, три пачки табака и 
$9$ коробок непромокаемых спичек, цену которых он не знал. Бармен потребовал $11$ долларов $80$ центов, на что 
Джо вытащил револьвер. Бармен сосчитал снова и исправил ошибку. Как Джо догадался, что бармен пытался его обсчитать?}

\zad{Коля и Петя купили одинаковые беговые лыжи. Сколько стоит пара лыж, если Петя уплатил стоимость лыж $3$-х рублевыми ассигнациями, 
Коля -- $5$-ти рублевыми, а всего они дали в кассу меньше $15$ купюр?}

\zad{а) $a + 2$ делится на $5$. Докажите, что $7a + 4$ делится на $5$.

б) $2000 + a$ и $999 - b$ делятся на $11$. Докажите, что $a + b$ делится на $11$.}

\zad{Докажите 

а) если сумма любых двух из трех чисел делится на три, то и сумма всех трех чисел делится на $3$;

б) если сумма любых трех из четырех чисел делится на $4$, то и каждое число делится на $4$;}

\zad{Докажите, что число, составленное из пятидесяти единиц, является составным.}

\zad{Докажите, что числа вида $aaa$ делятся и на $3$ и на $37$.}

\zad{На новом супер-калькуляторе есть только три кнопки : «умножить на $7$» , «прибавить $27$» и «вычесть $12$». 
Можно ли на этом калькуляторе из числа $6$ получить число $1$?}


\newpage

\begin{center}
  {\bf Серия 32. Разнобой 4.}
  
  {\small\bf 5 февраля}
\end{center}\vspace{5mm}

\resetz

\zad{Найдите все такие простые числа $p$, что $p^2 + 3$ тоже простое число.}

\zad{В таблице $6$ строк и $8$ столбцов. В некоторых клетках стоят звездочки. В каждой строке стоит не больше $3$ звездочек, 
а в каждом столбце -- не меньше двух. Сколько звездочек в таблице? Укажите все варианты.}

\zad{За последние три месяца Амир, Наум и Илья решали 60 задачи и получили за это 100 плюсиков. Каждая задача была кем-то решена.
Назовем задачу трудной, если ее решил один школьник. Назовем задачу легкой, если ее решили все трое. Докажите, что трудных задач 
на $20$ больше чем легких.}

\zad{В какое наибольшее число цветов можно раскрасить клетки доски $4 \times 4$ так, чтобы в каждой квадрате $2 \times 2$ нашлась пара клеток одного цвета?}

\zad{Найдите все такие простые числа $p$, что $p + 2$ и $p + 4$ тоже простые числа.}

\zad{100 Дедов Морозов помогали Снегурочке в доставке тяжелых сундуков с подарками (каждый сундук тащили 7 Дедов Морозов). 
Снегурочка подсчитала, что каждый Дед Мороз участвовал в переноске $57$ сундуков. Докажите, что она ошиблась.}

\zad{Найдите наименьшее возможное число членов кружка, если известно, что девочек в нем меньше $50\%$, но больше $40\%$?}


\newpage

\begin{center}
  {\bf Серия 33. Комбинаторика 4.}
  
  {\small\bf 9 февраля}
\end{center}\vspace{5mm}

\resetz



\zad{В правление избрано $9$ человек. Из них надо выбрать председателя, заместителя
председателя и секретаря. Сколькими способами это можно сделать?}

\zad{Сколько словарей надо издать, чтобы можно было непосредственно выполнять переводы с любого из 
пяти языков: русского, английского, французского, немецкого, итальянского на любой другой из этих пяти языков? На сколько больше словарей
придется издать, если число различных языков равно 10?}

\zad{У отца есть $5$ различных апельсинов, которые он дает своим $8$ сыновьям, причем каждый получает 
или один апельсин, или ничего. Сколькими способами это можно сделать? А если число апельсинов, получаемых каждым сыном, не ограничено?}

\zad{Сколько всего чисел от 1 до 1000 а) делится и на 3 и на 5; б) не делится ни на 5, ни на 7; в) делится либо на 3, либо на 7; 
г) не делится ни на 3, ни на 5, ни на 7; д) делится либо на 3, либо на 5, либо на 7?}

\zad{В соревновании по гимнастике участвуют 10 человек. Трое судей должны независимо друг от друга пронумеровать 
их в порядке, отражающем их выступление в соревновании. Победителем считается тот, кого назовут первым хотя бы двое судей.
В какой доле случаев победитель соревнований будет определен?}

\zad{Сколькими способами можно выбрать из полной колоды карт, содержащей 52 карты, по одной карте каждой масти? А 
если среди вынутых карт нет ни одной пары одинаковых, т. е. двух королей, двух десяток и т. д.?}


\newpage

\begin{center}
  {\bf Серия 34. Задачи на движения.}
  
  {\small\bf 12 февряля}
\end{center}\vspace{5mm}

\resetz


\zad{От потолка комнаты вертикально вниз по стене поползли две мухи. Спустившись до пола, они поползли обратно. 
Первая муха ползла в оба конца с одной и той же скоростью, а вторая хотя и поднималась вдвое медленнее первой, 
но зато спускалась вдвое быстрее. Какая из мух раньше приползет обратно?}

\zad{Я иду от дома до школы $30$ минут, а мой брат -- $40$ минут. Через сколько минут я догоню брата, если он вышел из дома на $5$ минут раньше меня?}

\zad{Дорога от дома до школы занимает у Пети $20$ минут. Однажды по дороге в школу он вспомнил, что забыл дома ручку. 
Если теперь он продолжит свой путь с той же скоростью, то придет в школу за $3$ минуты до звонка, а если вернется домой за ручкой, 
то, идя с той же скоростью, опоздает к началу урока на $7$ минут. Какую часть пути он прошел до того, как вспомнил о ручке? }

\zad{Мимо наблюдателя поезд проходит за $10$ секунд, а мимо моста длиной $400$ метров -- за $30$ секунд. Считается, что поезд проходит мимо 
моста начиная с того момента, когда локомотив въезжает на мост, и кончая моментом, когда последний вагон покидает мост. Определите длину и скорость поезда.}

\zad{Пловец плывёт вверх против течения Невы. Возле Дворцового моста он потерял пустую фляжку. Проплыв еще $20$ минут против течения, 
он заметил потерю и вернулся догонять флягу; догнал он её возле моста лейтенанта Шмидта. Какова скорость течения Невы, если расстояние между мостами равно $2$ км?}

\zad{Саша и Маша ехали на велосипеде друг навстречу другу: Саша из Сашкино в Машкино, Маша -- из Машкино в Сашкино. Они встретились, 
когда Саша проехал 12 км и еще треть оставшегося ему до Машкино пути, а Маша проехала $21$ км и еще четверть оставшегося ей до Сашкино пути.
Какое расстояние между Сашкино и Машкино?}

\zad{Ровно в $20$:$16$ два муравья начали ползти по дорожке навстречу друг другу. Они встретились, когда первый муравей прополз ровно 
треть всей дорожки. На следующий день первый муравей начал ползти по той же дорожке в $20$:$15$, а второй навстречу ему в $20$:$17$, 
и они встретились, когда первый муравей прополз половину дорожки. Какую часть всей дорожки успеет проползти до встречи первый муравей, 
если на третий день он начнёт ползти в $20$:$16$, а второй навстречу ему в $20$:$15$? }


\newpage

\begin{center}
  {\bf Серия 34. Разнобой 5.}
  
  {\small\bf 16 февраля}
\end{center}\vspace{5mm}

\resetz


\zad{Два пирата играли на золотые монеты. Сначала первый проиграл половину своих денег второму, потом второй проиграл 
половину своих денег первому, потом снова первый проиграл половину своих денег второму. В результате, у первого оказалось
15 монет, у второго -- 33. Сколько монет было у первого пирата до игры?}

\zad{Трем братьям дали 24 бублика так, что каждый получил на три бублика меньше, чем ему лет. Меньший брат был сообразительный 
и предложил поменять часть бубликов: «Я оставлю себе половину бубликов, а другую разделю между вами поровну, после этого средний 
брат поступит так же, а старший брат в конце поделит свои бублики как и мы.» После такого обмена все получили поровну. Сколько 
лет каждому брату?}

\zad{Из числа вычли сумму его цифр. Из полученного числа опять вычли сумму цифр полученного числа. После 11 таких вычитаний 
впервые получили «0». Какое число было в начале?}

\zad{Саша и Миша готовят подарки для девочек на 8 Марта. Саша в подарки кладёт на две конфеты больше, чем Миша. Зато Миша 
сделал на два подарка больше, чем Саша. Могло ли оказаться, что Миша и Саша вместе разложили 2003 конфеты, если в подарки 
они всегда кладут одинаковое количество конфет?}

\zad{Вся семья выпила по полной чашке кофе с молоком, причём Катя выпила четверть всего молока и шестую часть всего кофе. Сколько человек в семье?}

\zad{С числами можно выполнять следующие операции: умножать на два или произвольным образом переставлять цифры 
(нельзя только ставить нуль на первое место). Можно ли из $9$ получить $11111111$?}

\zad{Миша заметил, что на электронном табло, показывающем курс доллара к рублю (4 цифры, разделенные десятичной запятой), 
горят те же самые четыре различные цифры, что и месяц назад, но в другом порядке. При этом курс вырос ровно на $20\%$. 
Приведите пример того, как такое могло произойти.}


\newpage

\begin{center}
  {\bf Серия 35. Графы 3.}
  
  {\small\bf 19 февраля}
\end{center}\vspace{5mm}

\resetz


{\bf Определение:} Двудольный граф -- граф, вершины которого можно разбить на два множества так, что каждое ребро соединяет вершины из разных множеств.
Часто в контексте двудольных графов используется понятие цвета вершины. Разбитие графа на две доли называется покраской его вершин в два различных цвета 
(черный и белый).

\vspace{5mm}

\zad{В государстве $100$ городов, и из каждого из них выходит $10$ дорог. Сколько всего дорог в государстве?}

\zad{Может ли в государстве, в котором из каждого города выходит $11$ дорог, быть ровно $100$ дорог?}

\zad{В государстве $100$ обычных городов и одна столица. Наум утверждает, что из каждого обычного города выходит $10$ дорог, а из столицы -- $99$ дорог. 
Докажите, что Наум ошибается.}

\zad{Каково наибольшее возможное число рёбер

а) в графе с $n$ вершинами;

б) в двудольном графе с $b$ белыми и $r$ чёрными вершинами?}

\zad{В двудольном графе $100$ белых вершин и несколько чёрных. Степень каждой белой вершины равна $4$, а чёрной -- $5$. 
Сколько в этом графе а) рёбер; б) чёрных вершин?}

\zad{В стране Семёрка $15$ городов, каждый из которых соединён дорогами не менее, чем с $7$ другими. Докажите, что из 
каждого города можно добраться до любого другого (возможно, проезжая через другие города).}


\newpage

\begin{center}
  {\bf Серия 36. Введение переменной 3.}
  
  {\small\bf 5 марта}
\end{center}\vspace{5mm}
\resetz

\zad{Палиндром -- это число читающееся слева направо и справа налево одинаково (например, 5775 или 83438). Сергей написал 
трехзначный палиндром. Роберт прибавил к нему 42 и получил четырехзначный палиндром. Какие числа они написали.}

\zad{Килограмм говядины с костями стоит $78$ рублей, килограмм говядины без костей -- $90$ рублей, а килограмм костей -- $15$ рублей. Сколько
граммов костей в килограмме говядины?}

\zad{Велосипедист поднимался на холм со скоростью 12 км/час, а спустился он с холма тем же путем со скоростью 20 км/час, потратив на
спуск на 16 минут меньше, чем на подъем. Чему равна длина дороги,
ведущей на холм?}


\zad{Вася разрезал прямоугольник на два, сумма периметров которых равна 100. Петя разрезал такой же на два прямоугольника, 
сумма периметров которых равна 140. Чему мог быть равен периметр исходного прямоугольника?}

\zad{При сложении двух чисел Аделя пропустила нуль на конце одного слагаемого и получила сумму $2013$ вместо правильной суммы $3012$. 
Какие числа она должна была сложить?}

\zad{Три ковбоя зашли в салун. Один купил 4 сандвича, чашку кофе и 10 пончиков -- всего за 1 доллар и 69 центов. 
Второй купил 3 сандвича, чашку кофе и 7 пончиков за 1 доллар 26 центов. Сколько заплатил третий ковбой за один сандвич, чашку кофе и пончик?}

\zad{Найдите все трехзначные числа в 18 раз больше суммы своих цифр.}

\end{document}