\chapter{НАТУРАЛЬНЫЕ, ЦЕЛЫЕ, ПРОСТЫЕ И СОСТАВНЫЕ ЧИСЛА. ДЕЛИМОСТЬ.}

Натуральные, целые, простые и составные числа, признаки делимости чисел изучаются еще в младших классах. 
В кружковой работе, на факультативных занятиях как внеурочный материал можно изучать следующий дополнительный материал :

1. {\bf Основная теорема арифметики:} каждое натуральное число ${n > 1}$ единственным образом представимо в виде 
\[n = p_1^{\alpha_1}\cdot p_2^{\alpha_2}\cdot \ldots \cdot p_k^{\alpha_k},\] где $p_1 < p_2 < \ldots < p_k$ -- простые числа, и 
$\alpha_1, \alpha_2, \ldots, \alpha_k$ -- некоторые натуральные числа. Такое представление числа $n$ называется его {\it каноническим 
разложением} на простые сомножители.

2. Если натуральное число есть точный квадрат, то в его разложение каждый простой сомножитель входит в четной степени. Действительно, 
пусть ${n = m^2}$, где $n$ и $m$ -- натуральные числа. Из основной теоремы арифметики следует, что существует каноническое разложение
\[m = p_1^{\alpha_1}\cdot p_2^{\alpha_2} \cdot \ldots \cdot p_k^{\alpha_k}.\] Следовательно, \[n = m^2 = \left(p_1^{\alpha_1}\cdot p_2^{\alpha_2} 
\cdot \ldots \cdot p_k^{\alpha_k}\right)^2 = p_1^{2\alpha_1}\cdot p_2^{2\alpha_2} \cdot \ldots \cdot p_k^{2\alpha_k}.\]

3. Если натуральное число есть точный квадрат, то число делителей его нечетно; если не точный квадрат -- четное число. 
Это не трудно доказать: если число не точный квадрат, то каждому делителю найдется пара, значит число делителей 
есть четное число. А если число есть точный квадрат, одному делителю не найдется пара, отличная от него.

4. Точный квадрат никогда не оканчивается цифрами $2$, $3$, $7$, $8$.

5. Пусть каноническое разложение $n = p_1^{\alpha_1}\cdot p_2^{\alpha_2}\cdot \ldots \cdot p_k^{\alpha_k}$. Тогда число 
делителей числа $n$ вычисляется по формуле: \[\tau(n) = (\alpha_1 + 1)\cdot (\alpha_2 + 1)\cdot \ldots \cdot(\alpha_k + 1).\]

6. Среди последовательных трех натуральных чисел есть хотя бы одно число, делящееся на $3$ (в дальнейшем под словом “делимость” будем 
понимать “делимость” без остатка), а также, хотя бы одно четное число.

7. Как следствие признака $6$, можем сказать, что произведение трех последовательных чисел всегда делится на $6$.

8. Произведение двух последовательных четных чисел делится на $8$ без остатка.

9. При делении суммы цифр натурального числа на $3$ получается такой же остаток, как и при делении данного натурального числа на $3$.
Этот признак получается из признака делимости натурального числа на $3$.


